\newpage
\setcounter{section}{0}
\renewcommand{\thesection}{\arabic{section}}

\begin{center}
    \Huge
    \textbf{Modul 4}
    
    Konfigurasi VPN(Virtual Private Network) PPTP pada Mikrotik

\end{center}


\section{pendahuluan}


\section{Tujuan Praktikum}

Mengetahui cara menggunakan dan mengkonfigurasi VPN PPTP pada router mikrotik.

\section{Alat dan Bahan}

Berikut adalah Alat dan Bahan untuk praktikum:

\begin{enumerate}
    \item 2 Cloud Core Router
    \item 3 Kabel UTP (LAN)
    \item 3 Laptop
    \item Software Winbox
\end{enumerate}

\section{Topologi}


\section{Langkah Percobaan}
\begin{enumerate}
    \item Sambungkan PC dan router mikrotik sesuai dengan topologi
    \item Matikan firewall di laptop
    \item Masuk ke aplikasi Winbox
    \item Pada bagian Neighbour, check apakah ada IP 0000 identity mikrotik
    \item Reset mikrotik ke 0000
    \item Lalu tekan connect
    \item Lakukan konfigurasi DHCP agar dapat terhubung dengan ISP, pilih menu IP \texttt{\text>} DHCP Client \texttt{\text>} (+) \texttt{\text>} Interface : ether 1 (yang terhubung pada ISP)
    \item Kemudian secara otomatis akan didapatkan IP dari ISP
    \item Lalu pilih menu IP \texttt{\text>} Firewall \texttt{\text>} NAT \texttt{\text>} Chain : srcnat, Out. Interface : ether 1
    \item Kemudian pilih menu IP \texttt{\text>} Firewall \texttt{\text>} NAT \texttt{\text>} Action : masquerade
    \item Setelah itu atur routes untuk ether 1 secara static, pilih menu IP \texttt{\text>} Routes \texttt{\text>} (+) \texttt{\text>} Dst Address : 0.0.0.0/0, Gateway : (gateway IP address yang telah diberikan ISP) \texttt{\text>} Apply
    \item Setelah terlihat status “reachable” pada Route List, kemudian atur DNS
    \item Untuk mengaktifkan PPTP server, pilih menu PPP \texttt{\text>} Interface \texttt{\text>} PPTP Server \texttt{\text>} Default Profile : default encryption
    \item Kemudian buatlah secret untuk mengakses server, pilih menu New PPP Secret \texttt{\text>} Profile : default encryption , Local Address : (address PPTP server) , Remote Address : (IP yang akan diberikan ke client)
    \item Isikan nama dan password, pastikan nama dan password mudah untuk diingat
    \item Lalu lakukan konfigurasi client PPTP, pilih menu PPTP Client \texttt{\text>} New Interface \texttt{\text>} Connect To : (IP public server yang dituju)
    \item Kemudian pada kolom User dan Password, masukkan nama dan password sesuai secret yang sudah dibuat
    \item Setelah itu lakukan static routing, pilih menu New Route \texttt{\text>} Dst. Address : (jaringan local router lawan) , Gateway : (IP PPTP tunnel pada router lawan)
    \item Untuk melakukan remote client, perlu dibuat secret baru dengan cara yang sama dengan sebelumnya
    \item Agar remote client dapat terhubung ke server, perlu dilakukan setup connection pada sisi client
    \item Pergi ke setting dan pilih menu Network and Sharing Center \texttt{\text>} Set up new connection or network \texttt{\text>} Connect to a workplace \texttt{\text>} Use My Internet Connection (VPN) \texttt{\text>} Internet address : (IP public server yang dituju) \texttt{\text>} Next
    \item Kemudian masukkan nama dan password sesuai secret yang sudah dibuat untuk remote client
    
\end{enumerate}

\section{Hasil Percobaan}


\section{Kesimpulan}


\section{Tugas modul}

\begin{enumerate}
    \item 
\end{enumerate}